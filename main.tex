%\iffalse
\let\negmedspace\undefined
\let\negthickspace\undefined
\documentclass[journal,12pt,twocolumn]{IEEEtran}
\usepackage{cite}
\usepackage{circuitikz}
\usepackage{amsmath,amssymb,amsfonts,amsthm}
\usepackage{algorithmic}
\usepackage{graphicx}
\usepackage{textcomp}
\usepackage{xcolor}
\usepackage{txfonts}
\usepackage{listings}
\usepackage{enumitem}
\usepackage{mathtools}
\usepackage{gensymb}
\usepackage{comment}
\usepackage[breaklinks=true]{hyperref}
\usepackage{tkz-euclide} 
\usepackage{listings}
\usepackage{gvv}                                        
\def\inputGnumericTable{}                                 
\usepackage[latin1]{inputenc}                                
\usepackage{color}                                            
\newtheorem{theorem}{Theorem}[section]
\usepackage{array}                                            
\usepackage{longtable}                                       
\usepackage{calc}                                             
\usepackage{multirow}                                         
\usepackage{hhline}                                           
\usepackage{ifthen}                                           
\usepackage{lscape}
\usepackage{multicol}
\newtheorem{problem}{Problem}
\newtheorem{proposition}{Proposition}[section]
\newtheorem{lemma}{Lemma}[section]
\newtheorem{corollary}[theorem]{Corollary}
\newtheorem{example}{Example}[section]
\newtheorem{definition}[problem]{Definition}
\newcommand{\BEQA}{\begin{eqnarray}}
\newcommand{\EEQA}{\end{eqnarray}}
\newcommand{\define}{\stackrel{\triangle}{=}}
\theoremstyle{remark}
\newtheorem{rem}{Remark}
\begin{document}
\bibliographystyle{IEEEtran}
\vspace{3cm}
\title{Quadratic Equations and Inequations(Inequalities)}
\author{ai24btech11024 PRAHLADHA% <-this % stops a space
}
\maketitle
\newpage
\bigskip
\renewcommand{\thefigure}{\theenumi}
\renewcommand{\thetable}{\theenumi}
\maketitle\Large{Section D.MCQs with One or More than One Correct}
\begin{enumerate}[start=13]\large
 \item Number of integral divisors of the form $4n+2$\brak{n\geq 0} of the integer 240 is        \hfill (1984-2Marks)
 \begin{multicols}{2}
 \begin{enumerate}
     \item a positive integer
     \item divisible by n
     \item equal to n+$\frac{1}{n}$
     \item never equal to n
 \end{enumerate}
 \end{multicols}
 \item If $3^X=4^x-1$,then $x=$ \hfill (JEE Adv. 2013)
 \begin{multicols}{2}
 \begin{enumerate}
     \item $\frac{2\log_3 2}{2\log_3 2-1}$
     \item $\frac{2}{2-\log_2 3}$
     \item $\frac{1}{1-\log_4 3}$
     \item $\frac{2\log_2 3}{2\log_2 3-1}$
 \end{enumerate}
 \end{multicols}
 \item Let S be the set of all non-zero real numbers \(\alpha\) such that quadratic equation $\alpha x^2-x+\alpha=0$ has two distinct real roots $x_1$ and $x_2$ satisfying the inequality$|x_1-x_2|<1$. Which of the following intervals is(are) $\alpha$ subset of S? \hfill (JEE Adv. 2015)
 \begin{multicols}{2}
 \begin{enumerate}
     \item \brak{-\frac{1}{2},-\frac{1}{\sqrt{5}}}
     \item \brak{-\frac{1}{\sqrt{5}},0}
     \item \brak{0,\frac{1}{\sqrt{5}}}
    \item \brak{\frac{1}{\sqrt{5}},\frac{1}{2}}
 \end{enumerate}
 \end{multicols}
 \end{enumerate}

\maketitle\Large{Section E.Subjective Problems}\large
 \begin{enumerate}
     
 \item solve for $x$ : $4^x-3^{x-\frac{1}{2}}=3^{x+\frac{1}{2}}-2^{2x-1}$                       \hfill  (1978)

 

\item If \brak{m,n}=$\frac{\brak{1-x^m}\brak{1-x^{m-1}}.......\brak{1-x^{m-n+1}}}{\brak{1-x}\brak{1-x^2}.......\brak{1-x^n}}$

Where m and n are positive integers \brak{n \leq m}.show that 
$\brak{m,n+1}=\brak{m-1,n+1}+x^{m-n+1}\brak{m-1,n}$ \hfill (1978)
 \item Solve for x:$\sqrt{x+1}-\sqrt{x-1}=1$.   \hfill (1978)
 \item Solve the following equation for x:
 
$2\log_x a+\log_{ax} a+3\log_{a^2x} a=0$,$a>0$ \hfill (1978)
\item Show that the square of $\frac{\sqrt{26-15\sqrt{3}}}{5\sqrt{2}-\sqrt{38+5\sqrt{3}}}$ is a rational number.\hfill (1978) 
\item Sketch the solution set of the following system of inequalities:

$x^2+y^2-2x\geq 0;3x-y-12\leq 0;y-x\leq 0;y\geq 0$. \hfill (1978)
\item Find all integers x for which

$\brak{5x-1}<\brak{{x+1}}^2<\brak{7x-3}$. \hfill (1978)
\item If $\alpha,\beta$ are the roots of $x^2+px+q=0$ and $\gamma,\delta$ are the roots of $x^2+rx+s=0$,evaluate $\brak{\alpha-\gamma}\brak{\alpha-\delta}\brak{\beta-\gamma}\brak{\beta-\delta}$ in terms of p,q,r,and s. 

Deduce the condition that the equations have a common root. \hfill (1979)
\item Given $n^4<10^n$ for a fixed positive integer $n\geq 2$,

prove that $\brak{n+1}^4<10^{n+1}$. \hfill (1980)
\item Let $y=\sqrt{\frac{\brak{x+1}\brak{x-3}}{\brak{x-2}}}$

Find all the real values of x for which y takes real values. \hfill (1980)
\item For what values of m,does the system of equations \hfill (1980)

$3x+my=m$

$2x-5y=20$

has solution satisfying the condition $x>0,y>0$. \hfill (1980)
\item find the solution set of the system

$x+2y+z=1$;

$2x-3y-w=2$;

$x\geq 0;y\geq 0;z\geq 0;w\geq 0$.  \hfill (1980)

 \end{enumerate}
 

 

\end{document}

